\documentclass{problemset}

\renewcommand{\pset}{7}
%\renewcommand{\name}{Jane Doe}


\begin{document}

\begin{exercise} An element of a ring is called {\bf nilpotent} if \(a^n=0\) for some natural number \(n\).  Show that the set of nilpotent elements of a commutative ring form an ideal.
\end{exercise}

%\begin{solution}
%Solution goes here
%\end{solution}


\begin{exercise} 
Let \(R\) be a commutative ring, and \(I \subset R\) an ideal.  Show that the {\bf radical} of \(I\), defined by
\[ \sqrt{I} = \left\{ x \in R\ \middle|\ x^n \in I\ \text{for some}\ n \in \mathbb{N}\right\}\]
is an ideal.
\end{exercise}

%\begin{solution}
%Solution goes here
%\end{solution}

\begin{exercise} Let \(R\) be a commutative ring, and let \(X \subset R\) be a nonempty subset.  Show that the {\bf annihilator} of \(X\), defined by
\[\mathrm{Ann}(X)=\left\{ r \in R\ \middle|\ rx=0\ \text{for all}\ x\in X\right\}\]
is an ideal.
\end{exercise}

%\begin{solution}
%Solution goes here
%\end{solution}

\begin{exercise} Let \(f:R \rightarrow S\) be a ring homomorphism, and let \(I \subset R\) and \(J \subset S\) be ideals.
\begin{enumerate}[(a)]
\item Show that \(f^{-1}(J)\) is always an ideal that contains \(\ker f\).
\item Show that if \(f\) is surjective, then \(f(I)\) is an ideal in \(S\).
\end{enumerate}
\end{exercise}

%\begin{solution}
%Solution goes here
%\end{solution}


\begin{exercise} Show that every finite integral domain is a field.
\end{exercise}

%\begin{solution}
%Solution goes here
%\end{solution}

\begin{exercise} 
Prove the third isomorphism theorem for rings.
\end{exercise}

%\begin{solution}
%Solution goes here
%\end{solution}




%\begin{acknowledgements}
% Acknowledge your collaborators here. Be specific about who helped on which problems, and for whic parts.  For example, ``Jane helped me on Exercise 3. I got stuck proving X, but once she showed me the trick, I was able to finish.''
%\end{acknowledgements}

\end{document}