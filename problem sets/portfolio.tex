\documentclass{problemset}
\usepackage{titlesec,hyperref}

\renewenvironment{exercise}{\refstepcounter{exercise}
   \noindent \textbf{Exercise~\thesection.\theexercise. }}{ \vskip-\lastskip \vspace{0.2in} }
   
\titleformat{\section}[frame]
{\normalfont}
{}
{8pt}
{\Large\bfseries\filcenter}

\hypersetup{
  colorlinks   = true, %Colours links instead of ugly boxes
  urlcolor     = blue, %Colour for external hyperlinks
  linkcolor    = blue, %Colour of internal links
  citecolor   = red %Colour of citations
}


\chead{Portfolio}
%\renewcommand{\name}{Jane Doe}

\begin{document}

\tableofcontents
\newpage


\section{Problem Set 1}

\begin{exercise} Show the following directly from the definitions:
\begin{enumerate}[(a)]
\item If \(G\) is a monoid, the identity element is unique. 
\item If \(G\) is a group, every element has a unique inverse.
\end{enumerate}
\end{exercise}

\begin{solution}
%Solution goes here
\end{solution}

\begin{exercise}
Let \(G\) be a semigroup. Show that \(G\) is a group if and only if for all \(a,b \in G\), the equations \(ax=b\) and \(ya=b\) have solutions in \(G\).
\end{exercise}

\begin{solution}
%Solution goes here
\end{solution}

\begin{exercise}
Let \(G\) be a group.  Show that the following are equivalent:
\begin{enumerate}[(i)]
\item \(G\) is abelian.
\item \((ab)^2=a^2b^2\) for all \(a,b \in G\).
\item \((ab)^{-1}=a^{-1}b^{-1}\) for all \(a,b \in G\).
\item \((ab)^n=a^nb^n\) for three consecutive integers \(n\) and for all \(a,b \in G\).
\end{enumerate}
\end{exercise}

\begin{solution}
%Solution goes here
\end{solution}


\begin{exercise}
Let \(G\) be a semigroup. \(G\) is called \term{left cancellative} if \(ab=ac\) implies \(b=c\) (for all \(a,b,c \in G\)), and is called \term{right cancellative} if \(ba=ca\) implies \(b=c\) (again for all \(a,b,c \in G\).  A semigroup that is both left and right cancellative is just called \term{cancellative}. 
\begin{enumerate}[(a)]
\item Show that every finite cancellative semigroup is a group.
\item Give an example of an infinite cancellative semigroup that is not a group.
\end{enumerate}
\end{exercise}

\begin{solution}
%Solution goes here
\end{solution}

\begin{exercise}
Let \(G\) be a cyclic group. Show that every homomorphic image of \(G\) and every subgroup of \(G\) is also cyclic.
\end{exercise}

\begin{solution}
%Solution goes here
\end{solution}

\begin{exercise}
Let \(G\) be an abelian group of order \(pq\) for coprime \(p,q\in \mathbb{N}\).  Show that if \(G\) contains elements of both order \(p\) and order \(q\), then \(G\) must be cyclic.
\end{exercise}

\begin{solution}
%Solution goes here
\end{solution}

\begin{exercise}
Let \(f:G \rightarrow H\) be a group homomorphism.  Suppose \(a\in G\) such that \(f(a) \in H\) has finite order.  Show that if \(|a|\) is finite,  then \(|f(a)|\) divides \( |a|\).
\end{exercise}

\begin{solution}
%Solution goes here
\end{solution}

\begin{exercise}
Show that every group with a finite number of subgroups is itself finite.
\end{exercise}

\begin{solution}
%Solution goes here
\end{solution}


\begin{exercise}
Let \(H,K\) be subgroups of a group \(G\). Show that \(HK\) is a subgroup if and only if \(HK=KH\).
\end{exercise}

\begin{solution}
%Solution goes here
\end{solution}

\newpage
\section{Problem Set 2}


\begin{exercise} Let \(G\) be a group. Show that the following are equivalent:
\begin{enumerate}[(i)]
\item \(|G|\) is prime
\item \(G\) has exactly two subgroups (\(G\) and the trivial subgroup).
\item \(G \cong \mathbb{Z}_p\) for some prime \(p\).
\end{enumerate}
\end{exercise}

\begin{solution}
%Solution goes here
\end{solution}


\begin{exercise}Let \(G\) be a group, and let \(H\) and \(K\) subgroups of finite index. Show that if \([G:H]\) and \([G:K]\) are coprime, then \(G=HK\).
\end{exercise}

\begin{solution}
%Solution goes here
\end{solution}


\begin{exercise} 
Let \(H\), \(K\), and \(N\) be subgroups of a group \(G\). Show that if \(H < N\), then \(HK \cap N = H(K \cap N)\).
\end{exercise}

\begin{solution}
%Solution goes here
\end{solution}


\begin{exercise} 
Show that every subgroup of index 2 is normal.
\end{exercise}

\begin{solution}
%Solution goes here
\end{solution}


\begin{exercise} 
Let \(N = \left\{ \sigma \in S_4\ |\ \sigma(4)=4\right\}\).  Determine if \(N\) is a normal subgroup of \(S_4\) or not.
\end{exercise}

\begin{solution}
%Solution goes here
\end{solution}

\begin{exercise} 
Let \(Q = \langle i,j \ |\ i^4=e, i^2=j^2, iji=j \rangle\) (this is called the {\em quaternion group}).  Show that every subgroup of \(Q\) is normal.
\end{exercise}

\begin{solution}
%Solution goes here
\end{solution}

\newpage
\section{Problem Set 3}



\begin{exercise} If \(G\) is a group, the {\em center} of \(G\) is the group
 \[Z(G) = \left\{ a \in G\ |\ ax=xa \text{ for all } x\in G\right\}.\]
Show that the center is a normal subgroup of \(G\).
\end{exercise}

\begin{solution}
%Solution goes here
\end{solution}

\begin{exercise}Consider the subgroup \(N=\langle (123)\rangle\) of \(S_3\).
\begin{enumerate}[(a)]
\item Show that \(N\) is a normal subgroup.
\item Describe \(S_3/N\).
\end{enumerate}
\end{exercise}

\begin{solution}
%Solution goes here
\end{solution}


\begin{exercise}Let \(f: G \rightarrow H\) be a group homomorphism, and set \(N=\ker f\).  Let \(K<G\) be any subgroup.
\begin{enumerate}[(a)]
\item Show that \(f^{-1}\left(f(K)\right)=KN\).
\item Show that \(f^{-1}\left(f(K)\right)=K\) if and only if \(N < K\).
\end{enumerate}
\end{exercise}

\begin{solution}
%Solution goes here
\end{solution}

\begin{exercise}
Consider the subgroups \(H=\langle 7\rangle\) and \(K=\langle 42\rangle\) of \(\mathbb{Z}\).  Note that \(K \lhd H\);  describe \(H/K\). 
\end{exercise}

\begin{solution}
%Solution goes here
\end{solution}

\begin{exercise} Let \(G\) be a group. Show that if \(G/Z(G)\) is cyclic, then \(G\) is abelian.
\end{exercise}

\begin{solution}
%Solution goes here
\end{solution}

\newpage
\section{Problem Set 4}

\begin{exercise} Let \(f: G \rightarrow H\) be a group homomorphism and  suppose \(\ker f < N < G\) for some subgroup \(N\). Show that if \(H\) is abelian, then \(N\) must be normal.
\end{exercise}

\begin{solution}
%Solution goes here
\end{solution}


\begin{exercise} Consider the set \(N \leq S_4\) given by 
\[N=\left\{ (1), (12)(34), (13)(24), (14)(23)\right\}.\]
\begin{enumerate}[(a)]
\item Verify that \(N\) is a subgroup.
\item Show that \(N \lhd S_4\).
\item Conclude that \(A_4\) is not simple.
\end{enumerate}
\end{exercise}

\begin{solution}
%Solution goes here
\end{solution}

\begin{exercise}
Show that \(A_4\) has no subgroup of order 6.
\end{exercise}

\begin{solution}
%Solution goes here
\end{solution}


\begin{exercise}
Show that \(A_n\) is the only subgroup of \(S_n\) that has index 2 (\textit{Hint: show that an index 2 subgroup must contain a 3-cycle}).
\end{exercise}

\begin{solution}
%Solution goes here
\end{solution}

\begin{exercise} The dihedral group \(D_6\) and the alternating group \(A_4\) both have 12 elements.  Determine if they are isomorphic or not.
\end{exercise}

\begin{solution}
%Solution goes here
\end{solution}

\newpage
\section{Problem Set 5}

\begin{exercise} Let \(G\) be a group with two normal subgroups, \(N\) and \(H\).  Suppose that \(G = N \rtimes H\).
\begin{enumerate}[(a)]
\item Show that \(G = H \rtimes N\). 
\item Show that \(G = N \times H\).
\item Show that \(G\) is abelian if and only if \(N\) and \(H\) are abelian.
\end{enumerate}
\end{exercise}

\begin{solution}
%Solution goes here
\end{solution}

\begin{exercise} Let \(G, H\) be finite cyclic groups.  Show that \(G \times H\) is cyclic if and only if \(|G|\) and \(|H|\) are relatively prime.
\end{exercise}

\begin{solution}
%Solution goes here
\end{solution}

\begin{exercise} Show that \(S_3\) is \textbf{not} a direct product of any of its proper subgroups.
\end{exercise}

\begin{solution}
%Solution goes here
\end{solution}

\begin{exercise} Show that \(S_n\) is a semidirect product.
\end{exercise}

\begin{solution}
%Solution goes here
\end{solution}


\begin{exercise} Show that a free abelian group is a free group if and only if it is cyclic.
\end{exercise}

\begin{solution}
%Solution goes here
\end{solution}


\begin{exercise} Show that the direct sum of free abelian groups is a free abelian group (Note that this is not true for direct products).
\end{exercise}

\begin{solution}
%Solution goes here
\end{solution}

\newpage
\section{Problem Set 6}

\begin{exercise} Determine all of the Sylow \(3\)-subgroups of \(S_5\).
\end{exercise}

\begin{solution}
%Solution goes here
\end{solution}

\begin{exercise} Show that there is no simple group of order \(200\) (Hint: Look for a normal Sylow \(p\)-subgroup).
\end{exercise}

\begin{solution}
%Solution goes here
\end{solution}

\begin{exercise} 
Let \(G\) be a group of order \(p^nq\) for some primes \(p>q\). Show that \(G\) contains a unique normal subgroup of index \(q\).
\end{exercise}

\begin{solution}
%Solution goes here
\end{solution}

\begin{exercise} Show that every group of order \(p^2q\) for distinct primes \(p,q\) is solvable.
\end{exercise}

\begin{solution}
%Solution goes here
\end{solution}

\begin{exercise} 
Let \(G\) be a group, \(K \leq G\), and \(H \unlhd G\).  Show that if \(H\) and \(K\) are solvable, then \(HK\) is solvable as well.
\end{exercise}

\begin{solution}
%Solution goes here
\end{solution}

\begin{exercise} 
Use the Sylow theorems to show that every group of order 72 is solvable.  
\end{exercise}

\begin{solution}
%Solution goes here
\end{solution}

\newpage
\section{Problem Set 7}

\begin{exercise} An element of a ring is called {\bf nilpotent} if \(a^n=0\) for some natural number \(n\).  Show that the set of nilpotent elements of a commutative ring form an ideal.
\end{exercise}

\begin{solution}
%Solution goes here
\end{solution}


\begin{exercise} 
Let \(R\) be a commutative ring, and \(I \subset R\) an ideal.  Show that the {\bf radical} of \(I\), defined by
\[ \sqrt{I} = \left\{ x \in R\ \middle|\ x^n \in I\ \text{for some}\ n \in \mathbb{N}\right\}\]
is an ideal.
\end{exercise}

\begin{solution}
%Solution goes here
\end{solution}

\begin{exercise} Let \(R\) be a commutative ring, and let \(X \subset R\) be a nonempty subset.  Show that the {\bf annihilator} of \(X\), defined by
\[\mathrm{Ann}(X)=\left\{ r \in R\ \middle|\ rx=0\ \text{for all}\ x\in X\right\}\]
is an ideal.
\end{exercise}

\begin{solution}
%Solution goes here
\end{solution}

\begin{exercise} Let \(f:R \rightarrow S\) be a ring homomorphism, and let \(I \subset R\) and \(J \subset S\) be ideals.
\begin{enumerate}[(a)]
\item Show that \(f^{-1}(J)\) is always an ideal that contains \(\ker f\).
\item Show that if \(f\) is surjective, then \(f(I)\) is an ideal in \(S\).
\end{enumerate}
\end{exercise}

\begin{solution}
%Solution goes here
\end{solution}


\begin{exercise} Show that every finite integral domain is a field.
\end{exercise}

\begin{solution}
%Solution goes here
\end{solution}

\begin{exercise} 
Prove the third isomorphism theorem for rings.
\end{exercise}

\begin{solution}
%Solution goes here
\end{solution}

\begin{acknowledgements}
% Acknowledge your collaborators here. Be specific about who helped on which problems, and for whic parts.  For example, ``Jane helped me on Exercise 3. I got stuck proving X, but once she showed me the trick, I was able to finish.''
\end{acknowledgements}

\end{document}
