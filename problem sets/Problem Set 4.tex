\documentclass{problemset}

\renewcommand{\pset}{4}
%\renewcommand{\name}{Jane Doe}


\begin{document}

\begin{exercise} Let \(f: G \rightarrow H\) be a group homomorphism and  suppose \(\ker f < N < G\) for some subgroup \(N\). Show that if \(H\) is abelian, then \(N\) must be normal.
\end{exercise}

%\begin{solution}
%Solution goes here
%\end{solution}


\begin{exercise} Consider the set \(N \leq S_4\) given by 
\[N=\left\{ (1), (12)(34), (13)(24), (14)(23)\right\}.\]
\begin{enumerate}[(a)]
\item Verify that \(N\) is a subgroup.
\item Show that \(N \lhd S_4\).
\item Conclude that \(A_4\) is not simple.
\end{enumerate}
\end{exercise}

%\begin{solution}
%Solution goes here
%\end{solution}

\begin{exercise}
Show that \(A_4\) has no subgroup of order 6.
\end{exercise}

%\begin{solution}
%Solution goes here
%\end{solution}


\begin{exercise}
Show that \(A_n\) is the only subgroup of \(S_n\) that has index 2 (\textit{Hint: show that an index 2 subgroup must contain a 3-cycle}).
\end{exercise}

%\begin{solution}
%Solution goes here
%\end{solution}

\begin{exercise} The dihedral group \(D_6\) and the alternating group \(A_4\) both have 12 elements.  Determine if they are isomorphic or not.
\end{exercise}

%\begin{solution}
%Solution goes here
%\end{solution}





%\begin{acknowledgements}
% Acknowledge your collaborators here. Be specific about who helped on which problems, and for whic parts.  For example, ``Jane helped me on Exercise 3. I got stuck proving X, but once she showed me the trick, I was able to finish.''
%\end{acknowledgements}

\end{document}